\chapter{An introduction to Bayesian nonparametrics by
\href{http://sinead.github.io/}{Sinead Williamson}}

\section{The Dirichlet process}

\subsection{An urn representation}

\begin{itemize}
  \item Conditioned on $\phi$ \dots
\end{itemize}

\begin{align}
  p(z_i = k| z_{i:i-1} =& \int p(z_i = k| \pi) p(\pi|z_{1:i-1} d\pi \\
      =& \frac{\sum_{j=1}{i-1} \Indicator(z_j = k) + \alpha_k}{i - 1 + \sum_k
      \alpha_k}
\end{align}

\subsection{Exchangeability}

\begin{itemize}
  \item Changing the order of the observation does not change the probabilities:
    $p(r,g,g,r,b,r) = p(r,r,r,g,g,b)$
  \item This allows us to threat every data point as if it were the las one
    that we piked out
\end{itemize}

\subsection{Choosing the number of clusters}

\begin{itemize}
  \item The Dirichlet distribution is a great choice when there is a clear,
    fixed number of clusters\dots but sometimes that is not the case
  \item The finite mixture model had $K$ mixture components:
    \begin{equation}
      p(x_n|\pi, \{u_k\}, \{\Sigma_k\}) = \sum_{k=1}^K \pi_k \Normal(x_n|\mu_k,
      \sigma_k)
    \end{equation}
  \item To make sure we never run out of clusters, no matter how many data
    points we see, we need (countably) infinite clusters
    \begin{equation}
      p(x_n|\pi, \{u_k\}, \{\Sigma_k\}) = \sum_{k=1}^\infty \pi_k \Normal(x_n|\mu_k,
      \sigma_k)
    \end{equation}
\end{itemize}

\subsection{Constructing an appropriate prior}

\begin{itemize}
  \item Start off with w elements
    \begin{equation}
      \pi^{(2)} = (\pi_1^{(2)}, \pi_2^{(2)}) \from \text{Dirichlet}(\alpha/2,
      \alpha/2)
    \end{equation}
  \item Split each component according to our beta
    \begin{align}
      \pi^{(4)} = \dots \from \text{Dirichlet}(\alpha/4,
      \alpha/4,\alpha/4,\alpha/4)
    \end{align}
  \item Keep until infinity?
    \begin{align}
      \pi^{(K)} = \dots \from \text{Dirichlet}(\alpha/K,\dots,\alpha/K)
    \end{align}
\end{itemize}

See Ferguson 1973 \cite{ferguson1973bayesian}

\section{Dirichlet process and Dirichlet marginals}

\subsection{Conjugacy of the multinomial}

\begin{itemize}
  \item We saw that dirichlet distribution was a conjugate prior of the
    multinomial.
  \item This is also true for the Dirichlet process
  \item Pick a partition $A_1,\dots,A_k$ \dots
\end{itemize}

\subsection{The Chinese restaurant process}

\begin{itemize}
  \item We can describe a sample from a DP-distributed probability distribution
    in terms of the following restaurant metaphor
  \item Restaurant with infinitely many tables (serving different dish)
  \item The first person sits in the first empty table
  \item Second person sits at the first table with probability $1/(1+\alpha)$ or
    at a new table with $1/(1+\alpha)$.
  \item let $m_k$ be the number of people sat at the $k$th table. The $n$th
    customer sits at the $k$th table with probability. missing from here $m_k/(1 + \alpha)$

\end{itemize}

\subsection{The stick breaking construction}

See Sethuraman 1994 \cite{sethuraman1994constructive}.

Imagine a stick of unit lenght representing the total probability

\begin{enumerate}
  \item Sample a $Beta(1, \alpha)$ random variable $b_k$
  \item Break a fraction $b_k$ of the stick. This is the first atom.
  \item Sample a random location for this atom
  \item recurse on the remaining stick to get
  \item Repeat from $k = 1,2,\dots$
\end{enumerate}

\subsection{Indian Buffet Process}

\begin{enumerate}
  \item A customer enters a restaurant with an infinite number of dishes
  \item \dots
\end{enumerate}

\subsection{Bulding latent feature models using the IBP}

The number of latent features (apple, skull, thread, hat,\dots) can use a
Indian Buffet Process (IBP).

\begin{itemize}
  \item Unbounded number of latent features
  \item Each column of the IBP corresponds to one of an infinite number of
    features
  item Weach row of the IB<P selects a finite subset of these features
  \item The rich-get-richer property of the IBP ensures features are shared
  between data points
  \item We must pick a likelihood model that determines what the features look
    like and how they are combined
\end{itemize}

In order to do inference in the IBP

\begin{align}
  Z \from IBP(\alpha) \\
  A_k \from \Normal(0, \theta_A^2 \Identity) \\
  x_n \from \Normal(z_nA^T, \sigma_X^2\Identity)
\end{align}

\subsection{Summary}

\begin{itemize}
  \item The Dirichlet process is an infinite-dimensional analoge of the
    Dirichlet distribution
  \item we use the Dirichlet distribution for clustering data into $K$ clusters
  item similarly, we can use the Dirichlet process to cluster data into an
  unbounded (and growing) number of clusters
\item the indian buffet process is an infinite-dimensional model fro feature
  subset selection
\item we can use it to construct latent feature models with infinitely many
  features
\item we can customize the latent feature model to match our data
\item many more building blocks --gamma process, poisson process, pitman-yor
  process, kingman's coalescent
\item Next, we will take a look at some hierarchical models that use the DP and
  IBP as building blocks
\end{itemize}


\subsection{Latent Dirichlet allocation}

Dirichlet distributions are commonly used in topic models. These models
describe documents using a distribution over the ``topics'', where each topic
is a distribution over words (see Latent Dirichlet Allocation by Blei et al.
2003 \cite{blei2003latent})

\begin{enumerate}
  \item For each topic $k = 1, \dots, K$ sample a distribution over the words
    $\beta_k \from \text{Dirichlet}(\eta_1, \dots, \eta_V)$
\end{enumerate}

\subsection{Hierarchical Dirichlet process}

\begin{align}
  G_0 \from DP(\gamma, H) \\
  G_m \from DP(\alpha, G_0)
\end{align}

With small values of $\alpha$ the topics will be differnt, however with large
$\alpha$ all the models will be the same.

\subsection{The Chinese restaurant franchise}

\begin{enumerate}
  \item First restaurant (document)
    \item Customers pick tables acording to a Chines restaurant process with
      parameter $\alpha$
    \item Each table asks their waiter to pick a dish
    \item The waiter considers all the dishes that have been served previously
      in the franchise
\begin{itemize}
  \item Since it is the first restaurant the first table gets a new dish
  \item Second table gets the previous dish with probability $1/(1+\gamma)$ or
    a new otherwise
    \item Keep going like the previous example of a Chinese restaurant
\end{itemize}
  \item The second restaurant
  \item The costumers pick tables according to a Chinese restaurant process
    with parameter $\alpha$ (from all the possible tables seen in all the
    franchises)
  \item \dots
\end{enumerate}

\subsection{Basic network models: Erdös-Renyi models}

\section{Further resources}

\begin{itemize}
  \item A tutorial on bayesian nonparametric models, S.j. gershman and D.M.
    Beli
  \item The introduction of Erik Sudderth's PhD thesis
  \item Markov chain sampling methods for Dirichlet process mixture models, RM
    Neal, Journal of computational and graphical statistics, 2000
  \item Python: bnpy-dev, PyIBP
  \item Julia: BNP.jl
\end{itemize}
