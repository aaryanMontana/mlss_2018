\chapter{Reinforcement Learning by \href{http://www.jan-peters.net/}{Jan Peters}}

Fri. 14:30--16:00

\section{Optimal Control Systems}

Data to model to value function to policy back then to data.

\subsection{Markov Decision Problems}

A stationary MDP is defined as

\begin{itemize}
  \item a state space $s \in S$
  \item action space $a \in A$
  \item transition dynamics $P(s_{t+1}| s_t, a_t)$
  \item reward function $r(s,a)$
  \item initial state probabilities $\mu_0(s)$
\end{itemize}

The Markov property says that the transition dynamics depends only on the
current time step.

\begin{equation}
  P(s_{t+1}|s_t, a_t, s_{t-1}, a_{t-1}, \dots) = P(s_{t+1}|s_t, a_t)
\end{equation}


\subsection{Basic reinforcement learning loop}

The objective is to maximize the expected long-term reward

\begin{equation}
  J_\theta = \Expected_{\mu_0, P, \pi} [\sum_{t=1}^{T-1}\gamma^t r(s_t, a_t)]
\end{equation}

The algorithmic description of the value iteration

\begin{itemize}
  \item Init: $V_T^*(s) \leftarrow r_T(s), t=T$
    \begin{itemize}
      \item Compute Q-Function for time step t (for each state action pair)
        \begin{equation}
          Q_t^*(s,a) = r_t(s,a) + \gamma \sum_{s'} P_t (s'|s,a)V_{t+1}^*(s')
        \end{equation}
      \item Compute V-Function for time step t(for each state) (TODO, check if
        next equation is max or argmax)
        \begin{equation}
          V_t^*(s) = \max_a Q_t^*(s,a)
        \end{equation}
    \end{itemize}
  \item Repeat: $t=t-1$
  \item Until $t=1$
  \item REturn: Optimal policy for each time step
    \begin{equation}
      \pi_t^*(s) = \argmax_a Q_t^*(s,a)
    \end{equation}
\end{itemize}

The Bellman Equation (Bellman Principle of Optimality)

\begin{equation}
  V^*(s) = \max_a (r(s,a) + \gamma \Expected_p[V^*(s')|s,a)
\end{equation}


See Policy evaluation with temporal differences: a survey and comparison
\cite{dann2014policy}

When the max is expensive is possible to use the \emph{Policy Iteration}
method:

\begin{enumerate}
  \item Policy evaluation: Estimate quality of states (and actions) with
    current policy
  \item Policy improvement: Improve policy by taking actions with the highest
    quality
\end{enumerate}

\subsection{Linear Quadratic Gaussian Systems}

A Linear Quadratic Regulator (LQR) system is defined as

\begin{itemize}
  \item state space $x in \Real^n$
  \item action space $u in \Real^m$
  \item linear transition dynamics with Gaussian noise
    \begin{equation}
      p_t(x_{t+1}|x_t, u_t) = \Normal(x_{t+1}|A_t x_t + B_t u_t + b_t, \Sigma_t)
    \end{equation}
  \item quadratic reward function
    \begin{align}
      r_t(x,u) =& (x-r_t)^T R_t(x-r_t) + u_t^T H_t u_t \\
      r_t(x) =& (x-r_T)^T R_T(x-r_T)
    \end{align}
  \item initial state density
    \begin{equation}
      \mu_0(x) \Normal(x|\mu_0, \Sigma_0)
    \end{equation}
\end{itemize}

See Stefan Schaal, Christopher G. Atkeston 1998 \cite{schaal1998constructive}

The LQR systems need the initial point to be linearly related to the optimal
point (eg. keeping a stick balanced upwards). However, they can not solve
situations in which it is necessary a non-linear component (eg. if the stick
  starts from a hanging position, and it needs some sinuidal movement before it
can reach the upward position).

See work by Emo Todorov and Yuval Tassa on Incremental LQG (a simplification of
  Differential Dynamic Programming by Dyer and McReynolds 1969
\cite{dyer1969optimization})

With Optimal control we can compute optimal policies but only on

\begin{enumerate}
  \item Discrete Systems: (but world is not discrete)
  \item Linear Systems, Quadratic Reward, Gaussian Noise (LQR): (but the world
    is not linear)
  \item Along an optimal trajectory: (it is really hard to find)
\end{enumerate}

For these reasons we need to approximate.

\section{Value Function Methods}

Data to value function to policy to data.

One of the principles is that ``all models are wrong, but some are useful''. In
the previous section we created perfect models, but they may not be match the
real world. In that case, there can be drastical problems.

The \emph{Classical Reinforcement Learning} postulate is to solve the optimal
control problem by learning the value function and not the model.

\subsection{Markov Decision Processes (MDP)}

Infinite Horizon with a discounted reward parameter $\gamma$

Updates the value function based on samples $D = \{s_i, a_i, r_i, s_i'\}$

\subsection{Temporal difference learning}

We are incorporating an error value into the prediction of the states (see
  Reinforcement learning, rich sutton and andy barto, 1998
  \cite{sutton1998reinforcement}.

Withe our estimate we can compute the TD error and make a decision

\begin{itemize}
  \item if the estimation was higher, we decrease the prediction
  \item if the estimation was lower, we increase the prediction
\end{itemize}

\begin{enumerate}
  \item Init $V_0^*(s) \leftarrow 0$
  \item Repeat $t=t+1$
    \begin{itemize}
      \item Observe transition $(s_t,a_t,r_t,s_{t+1})$
      \item Compute TD error $\delta_t = r_t + \gamma V_t(s_{t+1}) - V_t(s_t)$
      \item Update V-Function $V_{t+1}(s_t) = V_t(s_t) + \alpha\delta_t$
    \end{itemize}
  \item until convergence of $V$
\end{enumerate}

But we do not want deterministic policies as these will not explore the space,
for that reason there are at least two policy for exploration

\begin{enumerate}
  \item Epsilon-Greedy Policy
  \item Soft-Max Policy (it has an important temperature parameter $\beta$)
\end{enumerate}

Update equations for learning the Q-function $Q(s,a)$

\begin{equation}
  Q_{t+1}(s_t,a_t) =  \dots
\end{equation}

In wich it is necessary to estimate the future action $a_?$. There are two
methods

\begin{enumerate}
  \item Q-learning: $a_? = \argmax_a Q_t(s_{t+1}, a)$
  \item SARSA: $a_? = a_{t+1}$, where $a_{t+1} \from \pi(a|s_{t+1})$
\end{enumerate}

\subsection{Approximating the value function}

Instead of creating the matrix $V$ we can approximate with any function
approximation method (see Dann et al: Policy evaluation with temporal
differences: a survey and comparison, JMLR, 2014 \cite{dann2014policy}).

Some remarks on temporal difference learning

\begin{itemize}
  \item It si not a proper stochastic gradient descent
  \item As the target values $V^\pi(s)$ change after each parameter update
  \item This ``ignoreance'' introduces a bias in our optimization
  \item \dots
\end{itemize}

\subsection{Batch-Mode Reinforcement Learning}

Online methods are typically data-inefficient as they use only once every
sample. The rehuse of the samples has been done in Least-squares temporal
difference learning and fitted q-iteration (Tree-Based batch mode reinforcement
learning \cite{ernst2005tree}, Batch reinforcement learning
\cite{lange2012batch}).

In Q-iteration we do as in Value-iteration, but we use Supervised Learning
methods to approximate the ? function.

See Reinforcement learning in robot soccer, 2009
\cite{riedmiller2009reinforcement}.


\section{Policy Search}

From data to policy and back to data.

see Reinforcement learning of motor skills with policy gradients, 2008
\cite{peters2008reinforcement}.

\subsection{Black-box approaches}


\begin{itemize}
  \item Perturb the parameters of your policy: $\delta J = J(\theta + \delta
    \theta) - J(\theta)$
  \item Approximate J by the first order Taylor approximation $J(\theta +
    \delta\theta) = J(\theta) + \frac{\partial
    J(\theta)}{\partial\theta}\delta\theta$
  \item Solve for $\frac{\partial J(\theta)}{\partial \theta}$ in a least
    squares sense (linear regresssion).
\end{itemize}

See a large classs of algorithms includes: Kiefer-Wolfowitch procedure,
Robbins-Monroe, Simultaneous Perturbation Stochastic Approximation SPSA,\dots

\subsection{Likelihood-Ratio Policy Gradient methods}

The expected long term reward $J(\theta)$ can be weritten as expectation over
the trajectory distribution.

\section{Key problems}

\begin{enumerate}
  \item no notion of data in the generic problem formulation
  \item optimization bias problematic with data
  \item role of tfeatures is unclear in most methods
\end{enumerate}
\bibliographystyle{apalike}
\bibliography{references}
