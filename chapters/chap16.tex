\chapter{Advances in Machine Learning for Molecules by
  \href{https://jmhl.org/}{José Miguel
Hernández-Lobato}}

Thu. 11:30--13:00

New molecules and materials can potentially solve important existing challenges
like drug and mediceine design for health care, energy production and storage,
and greenhouse gas conversion., energy production and storage, and greenhouse
gas conversion.

However, progress in drug and material discovery has been slow because of the
cost of colecting data amd making decision based on data, which require a lot
of human intervention.

Currently there are plenty of available datasets with the properties of real
and virtual molecules. It is also possible to simulate new molecules by
estimating their properties with \emph{density functional theory} (DFT) before
they are made in the laboratory.

Some example projects are \dots

\begin{mybox}
\textbf{\Large ``Robot scientist'' speeds up drug discovery}

\textbf{Automated AI lab that learns and formulates hypotheses has identified
promising anti-cancer and anti-malarial compounds}

An artificial intelligence system – or ‘robot scientist’ – capable of screening
potential drugs almost completely independently could speed up drug
development, say the UK researchers who created it. The approach has already
identified some promising leads, including an anti-cancer compound which also
shows anti-malarial properties.

The robot scientist – named ‘Eve’ – is actually a collection of machines
including several computers hooked up to the kind of automated instruments
already found in many labs. ‘The idea is to automate scientific research,’ says
lead author Ross King from the University of Manchester. ‘You tell the system
about the area of research you’re interested in … and then the computer has an
automated way of forming novel hypotheses about that area of science. It can
then design experiments to test these hypotheses and the lab robots go ahead
and actually do the experiments.’ The computer can also interpret the results,
modify its hypotheses and construct new tests completely autonomously, only
needing occasional human assistance to top up reagents and remove waste.

Eve’s predecessor – Adam – carried out genetic experiments in yeast and became
the first robotic system to independently make a scientific discovery. ‘What we
wanted to do with Eve was apply the same approach to something with more
immediate societal values,’ says King. Eve was designed specifically for drug
development and the team initially chose to focus on neglected tropical
diseases.

Many parts of the process, such as high throughput library screening, are
already highly automated, and Eve is kitted out with all the necessary
equipment to screen tens of thousands of compounds a day and identify leads.
But the system can also intelligently respond to results and create
standardised assays using synthetic biology. Once it has identified a lead
compound, Eve can engineer specific strains of yeast needed to screen for
activity against a particular disease. This makes it both quicker and cheaper
than standard drug screening methods, even those that already use automated
equipment.

The system already has proven successes, highlighting several drug candidates
that could be ‘repurposed’. For example, it showed the anti-cancer compound
TNP-470 can also attack the malarial parasite Plasmodium vivax by inhibiting an
essential enzyme. This compound is now being looked at in Brazil, where this
form of malaria is prevalent.

‘It was surprising to me how easy it was to find interesting compounds for
these diseases,’ says King. ‘When I started this I assumed that the main focus
of the research would be proof-of-principle.’

Andrey Rzhetsky, a geneticist and computational scientist at the University of
Chicago, US, who was not involved in the study, praised the group’s work. ‘I am
definitely a believer in this direction of AI [artificial intelligence] work,’
he comments.

In future, King says, robot scientists like Eve could be used by pharmaceutical
companies to streamline the drug development process, or explore potential new
functions for existing drugs. ‘We found some really interesting compounds,’ he
says, ‘And there are even more exciting results that we have yet to report.’

\textbf{References:}
K Williams et al, J. R. Soc., Interface, 2015, DOI: 10.1098/rsif.2014.1289
\textbf{Source:}
\href{https://www.chemistryworld.com/news/robot-scientist-speeds-up-drug-discovery/8230.article}{chemistryworld
by Emma Stoye}
\end{mybox}

Machine Learning (ML) can accelerate and automate the discovery process. If DFT can
be slow, it is possible to estimate the results with ML.

Some of the challenges is the fact that common ML methods only accept vectorial
input data, and molecules are commonl represented by graphs.

Some of the representations of molecules in machine learning are: (1) molecular
fingerprints, (2) SMILES, or (3) Graph neural networks (GNNs).

While the reaction prediction model can be achieved by \gls{GNN}-based, or seq2seq
methods.

\section{Molecular representation for ML}

\subsection{Molecular fingerprints}

An encoding of a molecular graph into a binary string (See Rogers et al. 2010
\cite{rogers2010extended})

The algorithm takes as an input a molecular graph with radious parameter $R$
and length $L$.

\begin{figure}[h]
  \centering
  \begin{tikzpicture}[x=0.5cm,y=0.5cm]
    \node[main node] (1) {$C$};
    \node[main node] (2) [right = 1.3 of 1]  {$R$};
    \node[main node] (3) [below left = 1.3 and 0.5 of 1]  {$O$};
    \node[main node] (4) [below right = 1.3 and 0.5 of 2] {$Z$};
    \node[main node] (5) [below right = 1.3 and 0.5 of 3]  {$H$};
    \node[main node] (6) [below left = 1.3 and 0.5 of 4] {$H$};

    \path[thick]
    (1) edge node {} (2)
    (1) edge node {} (3)
    (1) edge node {} (4)
    (2) edge node {} (4)
    (3) edge node {} (4)
    (3) edge node {} (5)
    (3) edge node {} (6);
  \end{tikzpicture}
\end{figure}

\begin{enumerate}
  \item Assign integers to atoms by applying has function to atom features
  \begin{figure}[h]
    \centering
    \begin{tikzpicture}[x=0.5cm,y=0.5cm]
      \node[main node] (1) {$1$};
      \node[main node] (2) [right = 1.3 of 1]  {$2$};
      \node[main node] (3) [below left = 1.3 and 0.5 of 1]  {$3$};
      \node[main node] (4) [below right = 1.3 and 0.5 of 2] {$4$};
      \node[main node] (5) [below right = 1.3 and 0.5 of 3]  {$5$};
      \node[main node] (6) [below left = 1.3 and 0.5 of 4] {$5$};

      \path[thick]
      (1) edge node {} (2)
      (1) edge node {} (3)
      (1) edge node {} (4)
      (2) edge node {} (4)
      (3) edge node {} (4)
      (3) edge node {} (5)
      (3) edge node {} (6);
    \end{tikzpicture}
  \end{figure}
  \item For $r=1,\dots,R$
    \begin{enumerate}
      \item Concatenate atom integers with integers of tneighboring atoms
      \begin{figure}[h]
        \centering
        \begin{tikzpicture}[x=0.5cm,y=0.5cm]
          \node[main node] (1) {$1,234$};
          \node[main node] (2) [right = 1.3 of 1]  {$2,14$};
          \node[main node] (3) [below left = 1.3 and 0.5 of 1]  {$3,1455$};
          \node[main node] (4) [below right = 1.3 and 0.5 of 2] {$4,123$};
          \node[main node] (5) [below right = 1.3 and 0.5 of 3]  {$5,3$};
          \node[main node] (6) [below left = 1.3 and 0.5 of 4] {$5,3$};

          \path[thick]
          (1) edge node {} (2)
          (1) edge node {} (3)
          (1) edge node {} (4)
          (2) edge node {} (4)
          (3) edge node {} (4)
          (3) edge node {} (5)
          (3) edge node {} (6);
        \end{tikzpicture}
      \end{figure}
      \item Assign new integers to atoms by applying has function to
        concatenation
      \begin{figure}[h]
        \centering
        \begin{tikzpicture}[x=0.5cm,y=0.5cm]
          \node[main node] (1) {$20$};
          \node[main node] (2) [right = 1.3 of 1]  {$45$};
          \node[main node] (3) [below left = 1.3 and 0.5 of 1]  {$13$};
          \node[main node] (4) [below right = 1.3 and 0.5 of 2] {$31$};
          \node[main node] (5) [below right = 1.3 and 0.5 of 3]  {$16$};
          \node[main node] (6) [below left = 1.3 and 0.5 of 4] {$16$};

          \path[thick]
          (1) edge node {} (2)
          (1) edge node {} (3)
          (1) edge node {} (4)
          (2) edge node {} (4)
          (3) edge node {} (4)
          (3) edge node {} (5)
          (3) edge node {} (6);
        \end{tikzpicture}
      \end{figure}
    \end{enumerate}
  \item Create $L$-dimensional zero vector $f$
  \item map generated integers to an entry in $f$ which is set to 1.
    (01110001010001001000100101)
\end{enumerate}


This method is fast to compute, in practice produces very good predictive
performance, and it is easy to interpret as the features represent the presence
of substructure.

On the other hand, the generated features are handcrafted and not data
dependent, and they are not smooth (similar segments may have different
representations).

\subsection{SMILES}

Simplified Molecular Input Line Entry System (SMILES) allows the representation
of a molecular graph in line notation. For example, $CC$ represents $CH_3CH_3$
(ethane), $CC(=O)O$ represents $CH_3COOH$ (acetic acid), $C1CCCCC1$ represents
$C_6H_{12}$ (cyclo\dots). See Sanchez-Lengleling and Aspuru-Guzik, 2018
\cite{gomez2018automatic}.

With the new sequence representations as stirngs it is possible to apply
Artificial Neural Networks. Some examples are Recurrent Neural Networks applied
to the string sequence to predict the following character, or use 1D
convolutional Neural Networks with lefat and right context.

With this method is easy to encode molecules as simple text strings, relatively
easy to understand by humans, and \acrfull{nlp} methods can be applied.

One of the disadvantages of SMILES is that the representation is not invariant
to the starting atom, and atoms close inthe graph may be far away in the string
representation.

\subsection{Graph Neural Networks (GNNs)}

This representation naturally encodes invariances to permutation of nodes, and
distances between atoms (See more about \glspl{GNN} in \cite{scarselli2009graph},
\cite{li2018learning}).

A \gls{GNN} includes the vecotrial variables

\begin{enumerate}
  \item $\{\vec{e}_{j\sim k}\}$ vecot of edges between nodes j and k
  \item $\{\vec{v}_i\}_{i=1}^N$ node features
  \item $\vec{u}$ are global features summarizing the graph properties
\end{enumerate}

\subsubsection{Set functions and auxiliary variables}

Set functions have as input sets of elements and as output a single element
summarizing the input set. It is invariant to input permuatation and accepts a
variable number of arguments. Some examples areelementwise summation, mean,
maximum, nimumum, \dots

The forward pass in a \gls{GNN} is

Given an initial set of vertices $v$, edges $e$ and global features $u$.

\begin{enumerate}
  \item For every edge
    \begin{enumerate}
      \item  updates the edge features
    \end{enumerate}
  \item For every vertice
    \begin{enumerate}
      \item summarize the incoming edge to $i$
      \item Update feature for node $i$
    \end{enumerate}
  \item Sumamrize all edges
  \item Summarize all nodes
  \item Update global features
  \item Compute prediction from features $\vec{u}$
  \item return $MLP(\vec{u})$
  \item Repeat all the steps several times $l=1,\dots,L$ (every step increases
    the range of molecules; it is necessary to set $L$ to capture the biggest
  structure in which we are interested).
\end{enumerate}

Specific implementations of \glspl{GNN}

\begin{itemize}
  \item Message passing neural network (MPNN) Gilmer et al, 2017
  \item Gated graph neural network (GGNN) Li et al. 2016
  \item \acrfull{wln}. Jin et al. 2017
  \item Neural graph fingerprints (NGFs). Duvenaud et al. 2015
\end{itemize}

Advantages of \glspl{GNN}
\bibliographystyle{apalike}
\bibliography{references}
